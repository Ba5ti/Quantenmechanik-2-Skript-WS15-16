	wobei 
		\begin{align*}
			\text{Vernichtnungsoperator:~} a &= \frac{1}{\sqrt{2}} 
			\left( y + i \tilde{p} \right) \\
			\text{Erzeugungsoperator:~} a^\dagger &= \frac{1}{\sqrt{2}} 
			\left( y - i \tilde{p} \right)
		\end{align*}
	Und da dann
		\begin{align*}
			[a , a^\dagger] &= 1 &\text{~oder~} a a^\dagger &= a^\dagger a + 1 \\
			\Rightarrow \tilde{H} = \left( a^\dagger a + \frac{1}{2} \right)& \\
			\text{und somit~} 
			y &= \frac{1}{\sqrt{2}} \left( a + a^\dagger \right),& 
			\tilde{p} &= \frac{1}{\sqrt{2}} \left( a - a^\dagger \right)
		\end{align*}
	Sei nun:
		\begin{align*}
			\tilde{H} \ket{n} &= \epsilon_n \ket{n} &\left( H \ket{n} = E_n \ket{n} 
			\Rightarrow E_n = \hbar \omega \epsilon_n \right)
			a^\dagger a \ket{n} = \left( \epsilon_n - \frac{1}{2} \right) \ket{n}
		\end{align*}
	(n ist Anzahl der Knoten (Nullstellen) der Wellenfunktionen.)
	Aus QM 1:
		\begin{align*}
			a \ket{n} &= \sqrt{n} \ket{n-1} \\
			a^\dagger \ket {n} &= \sqrt{n+1} \ket{n+1} \\
			&\Rightarrow \epsilon_{n+1} = \epsilon_n + 1
		\end{align*}
	mit $n = 0, 1, 2 \ldots$ 
		\begin{align*}
			&\epsilon_n = n + \frac{1}{2} \leadsto E_n = \hbar \omega \left(n + \frac{1}{2} \right) \\
			&\text{und~} \underbrace{a^\dagger a}_{\substack{N}}
			%\footnote{\text{Nummernoperator (Teilchenzahloperator)}}
			\ket{n} = n \ket{n}
		\end{align*} %ich hab die footnote nicht machen können
	Somit ist $\tilde{H} = N + \frac{1}{2}$-
	
\section{Zeitabhängige Störung}
%\numberwithin{equation}{chapter}
%\setcounter{equation}{1}
	\subsection{Zeitabhängige Störungstheorie und das Wechselwirkungsbild}
	Betrachte 
		\begin{equation*}
			H(t) = H^0 + H^1 (t)
		\end{equation*}
	$H^0$ ist zeitunabhängig: $H^1(t) = 0$ für $t < t_i$ oder $t > t_f$
	Ungestörtes System: 
		\begin{align*}
			H^0 \ket{n} &= E_n \ket{n} = \hbar \omega_n \ket{n} \\
			\ket{\psi(t)} &= \sum_n \braket{n | \psi(t_0)} e^{-i \omega_n (t-t_0)} \ket{n} \text{~für~} t, t_0 < t_i (\text{oder~} t, t_0 > t_f)
		\end{align*}
	Von nun an: $\ket{\psi(t_0)} = \ket{n}$ für $t_0 < t_i$
		\begin{equation*}
			\Rightarrow \boxed{\ket{\psi (t)} = e^{-i \omega_n (t-t_0)} \ket{n}
			} \text{~für~} t_0, t < t_i
		\end{equation*}		
	Formale Lösung von 
		\begin{equation*}
			i \hbar \frac{\diff}{\diff t} \ket{\psi(t)}
			= H^0 \ket{\psi(t)}
		\end{equation*}			
	Ist
		\begin{equation*}
			\Rightarrow \ket{\psi(t)} = \text{T} e^{-\frac{i}{\hbar} H^0 (t-t_0)} 
			\ket{\psi(t_0)} = \U^0(t-t_0) \ket{\psi(t_0)}
		\end{equation*}
	Achntung: T ist Zeit\textbf{ordnungs}operator und $\U^0(t-t_0)$ ist Zeit\textbf{entwicklungs}operator.
	Eigenschaften:
		\begin{align*}
			\left.
			\begin{aligned}
				\U^0(t_1) \U^0(t_2) &= \U^0(t_1 + t_2) \\
				\U^0(-t) &= \U^{0\dagger} (t)		
			\end{aligned}
			\right\} \U^{0\dagger} (t) \U^0 (t) = \mathds{1}
		\end{align*}
		\begin{align*}
			1 &= \braket{\psi(t) | \psi(t)} \\
			&= \braket{\psi(t_0) | \U^{0 \dagger} (t-t_0) \U^0 (t-t_0) | \psi(t_0)} \\
			&= \braket{\psi(t_0) | \psi(t_0)}
		\end{align*}
	Norm bleibt erhalten $\Rightarrow U^0$ ist unitär! \\
	Betrachte nun 
		\begin{equation*}
			i \hbar \frac{\diff}{\diff t} \ket{\psi(t)} = H(t) \ket{\psi(t)}
		\end{equation*}
	Endzustand ($t>t_f$):
		\begin{equation*}
			\boxed{ \ket{\psi(t)} = 
			\sum_m c_m(t) e^{-i \omega_m (t-t_0)} \ket{m}
			}
		\end{equation*}
	Übergangswahrscheinlichkeit von $\ket{n}$ in den Zustand $\ket{m}$:
		\begin{equation*}
			\boxed{P_{mn} (t) = |c_m(t)|^2}
		\end{equation*}
	Das gestörte System
		\begin{equation*}
			i \hbar \frac{\diff}{\diff t} \ket{\psi(t)}
			= \left( H^0 + H^1(t) \right) \ket{\psi(t)}
		\end{equation*}
	wird im Allgemeinen nicht durch
		\begin{equation*}
			\ket{\psi(t)} = \text{T} e^{-\frac{i}{\hbar} H(t) (t-t_0)} \ket{\psi(t_0}
		\end{equation*}
	gestört! $H(t)$ ist zeitabhängig.
	Aber 
		\begin{align*}
			\braket{\psi(t) | \psi(t)} &= \braket{\psi(t_0) | \psi(t_0)} \\
			\Rightarrow \ket{\psi(t)} &= \U(t-t_0) \ket{\psi(t_0)}
		\end{align*}
	$\U(t,t_0)$ ist ein noch unbekannter unitärer Zeitentwicklungsoperator.\\
	$\ket{\psi(t)}$ ist Wellenfunktion im Schrödingerbild.
	
	Betrachte: